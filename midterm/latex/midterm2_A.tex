\documentclass[a4paper, 11pt]{article}

% packages
\usepackage{kotex}
\usepackage{amssymb}
\usepackage{booktabs, caption}
\usepackage[flushleft]{threeparttable}
\usepackage{fullpage}
\usepackage{algorithm, algpseudocode, algorithmicx}

\title{Report of Question2-A}
\author{2019xxxxxx 박준영}
\date{}

\begin{document}
\maketitle

\section{Implementation}
분할정복(Divide and Conquer)을 이용해 해결했다. 전체 tree를 여러 개의 tree로 쪼개어 각 부분에서 주어진 범위에 있는 node의 개수를 각각 구해 전부 합치는 방식으로 구현했다.

\begin{algorithm}
	\caption{Range search}
	\begin{algorithmic}
		\Function{RangeSearch}{node, k1, k2}\Comment{k1 is lower bound, k2 is upper bound}
			\If{node is null}
				\State \Return{0}
			\EndIf\\
						
			\If{node's value $\in$ [k1, k2]}
				\State leftCount $\gets$ RangeSearch(node.leftChild, k1, k2)
				\State rightCount $\gets$ RangeSearch(node.rightChild, k1, k2)
				\State \Return{1 + leftCount + rightCount}
			\ElsIf{node's value $<$ k1}
				\State \Return{RangeSearch(node.rightChild, k1, k2)}
			\Else
				\State \Return{RangeSearch(node.leftChild, k1, k2)}
			\EndIf
		\EndFunction
	\end{algorithmic}
\end{algorithm}

\section{Time complexity}
Binary Search Tree에서 lower bound와 같거나 큰 node를 찾는 것은 O($h$)의 시간이 소요된다. (단, $h$는 BST의 height) 또한, tree를 순회하면서 node의 value가 $[k1,k2]$ 안에 있으면 count를 1 증가하는 것은 range의 안에 속하는 node 개수 $k$에 대하여 O($k$)의 시간이 소요된다. 따라서 위 알고리즘의 time complexity는 O($h+k$)

\end{document}
